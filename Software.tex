%
% Software.tex
% Free Software
%
% Freedom Sandwich Developer's Guide
%
% Copyright (C) 2014 Aleph Objects, Inc.
%
% This document is licensed under the Creative Commons Attribution 4.0
% International Public License (CC BY-SA 4.0) by Aleph Objects, Inc.
%
\section{Intro}
This chapter covers the software that runs on the Freedom Sandwich embedded hardware board.


\section{Bootloader}
\subsection{Intro}
U-boot is the bootloader. The linux-sunxi branch is used as a base.

\subsection{Git Repos}
Upstream git repos used:
\begin{itemize}
  \item{\url{git://github.com/linux-sunxi/u-boot-sunxi.git}}
  \item{\url{git://github.com/linux-sunxi/sunxi-tools.git}}
  \item{\url{git://github.com/linux-sunxi/sunxi-boards.git}}
  \item{\url{git://github.com/linux-sunxi/sunxi-bsp.git}}
  \item{We're not using it at present, but this is allegedly the best version of
  A20 u-boot at present, per
  
  \url{https://wiki.debian.org/InstallingDebianOn/Allwinner}
  
  \url{git://github.com/jwrdegoede/u-boot-sunxi.git}
  }
\end{itemize}

\subsection{Commands}
Various commands for reference.

\begin{itemize}
  \item{
  \begin{verbatim}
./sunxi-tools/fex2bin             \
OUTPUT/ao-pcduino3.fex OUTPUT/script.bin
  \end{verbatim}
  }

  \item{\begin{verbatim}
mkimage -C none -A arm -T script -d          \
OUTPUT/boot.cmd OUTPUT/boot/boot.scr
  \end{verbatim}
  }

  \item{
  \begin{verbatim}
sudo dd if=u-boot-sunxi-with-spl.bin of=/dev/sdb bs=1024 seek=8
  \end{verbatim}
  }

  \item{Boot splash, to boot.cmd:
  
  \verb|${fs}load ${dtype} ${disk}:1 10000002 /splash.bmp && bmp d 10000002 ;|
  }

  \item{
  \begin{verbatim}
make CROSS_COMPILE=arm-linux-gnueabihf-     \
Linksprite_pcDuino3_config
  \end{verbatim}
  }

  \item{
  \verb|make CROSS_COMPILE=arm-linux-gnueabihf-|
  }

\end{itemize}


\section{Linux Kernel}

\subsection{Intro}

The Freedom Sandwich board uses the Linux kernel.


\subsection{Kernel Branch}

We will be using Linux Sunxi as the base source code for the kernel.


There are various kernel branches that could be used as a base:

\begin{itemize}
  \item{Mainline Linus -- This doesn't have much of the latest/greatest for A20.

  \url{git://git.kernel.org/pub/scm/linux/kernel/git/torvalds/linux.git}}

  \item{Linksprite -- This is optimized for pcduino3, which is very similar to the
  board we'll be using. It uses non-free software, such as the Mali driver.
  Using it as a base will be a bit messy. It also does non-standard patching
  in the build process, instead of just committing everything to git. The
  build system overall is nice and does more than just the kernel. It depends
  upon the non-free Allwinner livesuit image tools. We will use another tree
  as a base, but will pick select drivers from this one.

  \url{git://github.com/pcduino/a20-kernel.git}}

  \item{Linux Sunxi -- This is the main kernel branch for the Sunxi platform built
  upon the Allwinner ARM chips. It is actively maintained. The Easy TAZ Mini
  core is built upon their various archives. Main website:

  \url{http://linux-sunxi.org/}

  \url{git://github.com/linux-sunxi/linux-sunxi.git}}

\end{itemize}



\subsection{Kernel Version}

We will be using the main linux-sunxi git repo, using the sunxi-3.4 branch as
the main base for the Linux kernel. The latest version is 3.4.90.

There are also various kernel versions we could chose from. Some options:

\begin{itemize}
  \item sunxi-next -- This looks pretty good. Worth exploring more.

  \item sunxi-devel -- Probably not.

  \item 3.4.79+ -- This is the kernel that gets built by the default a20-kernel
  archive from Linksprite. Known to work. Has non-free software.

  \item 3.14 -- This is the latest version from the sunxi-3.14 branch of the main
  linux-sunxi kernel. It has not seen as much development or testing as
  sunxi-3.4. It does have -sunxi patches and is based on a much more recent
  upstream kernel. The one test kernel I built didn't fully boot, but it
  likely can be made to work without too much pain. As it hasn't seen as much
  real-world usage, sunxi-3.4 is preferred.

  \item 3.4.90 -- This is the latest version from the sunxi-3.4 branch of the main
  linux-sunxi kernel. This is known to work. We will likely use a version of
  this kernel.
\end{itemize}

\subsection{Building the Kernel}


Quickie overview:

\begin{enumerate}
  \item{Clone the kernel archive we want to use:
  
  \verb|git clone| \url{git://github.com/linux-sunxi/linux-sunxi.git}
  
  \verb|cd linux-sunxi|
  }

  \item{Checkout the branch we want:
  
  \verb|git checkout sunxi-3.4|
  }

  \item{Copy over camillia kernel config (check for newer version):
  
  \verb|wget| \url{http://devel.lulzbot.com/Easy_TAZ_Mini/camellia/software/current/OUTPUT/eztaz_defconfig/eztaz_defconfig-ao11}
  
  \verb|cp eztaz_defconfig-ao11 linux-sunxi/.config|
  }

  \item{Go into kernel config and make whatever changes:

  \begin{verbatim}
LOADADDR=0x40008000 make -j1 ARCH=arm         \
CROSS_COMPILE=arm-linux-gnueabihf- menuconfig
  \end{verbatim}
  }

  \item{Copy that \verb|.config| somewhere as backup
  
  \verb|cp .config ../OUTPUT/eztaz_defconfig/eztaz_defconfig-ao999|
  }

  \item{Build the kernel uImage (no modules):
  
  \begin{verbatim}
LOADADDR=0x40008000 make -j4 ARCH=arm         \
CROSS_COMPILE=arm-linux-gnueabihf- uImage dtbs
  \end{verbatim}
  }

  \item{DEPRECATED, no modules now:
  Install kernel modules to the \texttt{OUTPUT} dir:

  \begin{verbatim}
LOADADDR=0x40008000 make -j4 ARCH=arm         \
CROSS_COMPILE=arm-linux-gnueabihf-            \
INSTALL_MOD_PATH=../OUTPUT modules_install
  \end{verbatim}
  }

\end{enumerate}


Misc build commands

\begin{itemize}
  \item{This will build the default sun7i (pcduino3) kernel (example):
  
  \begin{verbatim}
LOADADDR=0x40008000 make -j4 ARCH=arm           \
CROSS_COMPILE=arm-linux-gnueabihf-              \
sun7i_defconfig
  \end{verbatim}
  }

  \item{If you need to clean up:
  
  \begin{verbatim}
make clean ARCH=arm CROSS_COMPILE=arm-linux-gnueabihf-
  \end{verbatim}
  }

  \item{You probably don't need to:
  
  \begin{verbatim}
make mrproper ARCH=arm CROSS_COMPILE=arm-linux-gnueabihf-
  \end{verbatim}
  }
\end{itemize}


\section{Core OS}
\begin{itemize}
  \item{Debian Wheezy (stable) release}
  \item{armhf architecture}
\end{itemize}

\subsection{Creating Root Filesystem}
The root filesystem is Debian Wheezy using the armhf architecture.

Commands, briefly, to build a rootfs:

\begin{enumerate}
  \item{mkdir rootfs}
  \item{
  \begin{verbatim}
sudo debootstrap --verbose --arch=armhf --foreign wheezy  \
./rootfs
  \end{verbatim}
  }
% --include=alpha,beta to include specific packages

  \item{
  \begin{verbatim}
sudo cp -p /usr/bin/qemu-arm-static                      \
./rootfs/usr/bin/
  \end{verbatim}
  }

  \item{
\verb|sudo chroot ./rootfs|
  }

  \item{
\verb|/debootstrap/debootstrap --second-stage|
  }

\end{enumerate}


\subsection{Packages}
The core packages will be straight from Debian's armhf archive, including
wheezy-backports. Exceptions:

\begin{itemize}
  \item{u-boot bootloader} - Not maintained in a package (at present).
  \item{Linux Kernel} - Not maintained in a package (at present).
  \item{Slic3r} - This has been (?) built for Debian Sid and Jessie.
  \item{Slic3r dependencies} - This is a long list. They have been built for
  Debian Sid and Jessie.
  \item{OctoPrint} - This has not been packaged yet.
  \item{EZTAZ Applications} - Need to be packaged.
\end{itemize}

The development repository will be located here:

\url{http://devel.lulzbot.com/debian/}

The release repository will be located here:

\url{http://download.lulzbot.com/debian/}

So the development repo line to add to \verb|/etc/apt/sources.list|
for Azalea is:

\verb|deb http://devel.lulzbot.com/debian/ azalea main|


It will need to be populated with a sub-set of the main Debian archive. We
won't carry the whole repo, to make it more efficient. The main Debian repo
should remain compatible with the EZTAZ repo.


\subsection{Changes to Core Packages}
The following scripts/configurations to the standard Debian release:

\begin{itemize}
  \item{systemd} -- This should likely be used to improve boot time.
  \item{read-only filesystem} -- So the SD card won't corrupt itself, nor the
  user wait for fscking.
\end{itemize}


\subsection{Filesystem}
The best filesystem for the SD card needs to be selected... Considerations:

\begin{itemize}
  \item{Robustness}
  \item{Boot Speed}
\end{itemize}

Filesystem options for core OS:

\begin{itemize}
  \item{ext4}
  \item{btrfs}
  \item{squashfs}
  \item{NILFS2}
\end{itemize}

Filesystem options for users' USB drives:

\begin{itemize}
  \item{ext4}
  \item{btrfs}
  \item{FAT/VFAT} -- We won't use these in the core OS, but we will have to
  read them off users' USB drives.
  \item{NTFS} -- Same as FAT? Probably don't need/want it at all.
  \item{HFS+} -- Same as NTFS?
\end{itemize}


\subsection{Developer Host System Setup}

\begin{itemize}
  \item{Debian Wheezy}
  \item{emdebian repo} -- cross compilers
  \item{Fix compiler paths} -- cross compilers, use ccache
\end{itemize}

\subsubsection{Cross Compiler}
To compile ARM packages, a cross compiler must be set up.

\begin{enumerate}
  \item{Add emdebian Repo -- Add this line to \verb|/etc/apt/sources.list|:


   \verb|deb http://www.emdebian.org/debian/ unstable main|}
  \item{Update -- 

\verb|apt-get update|}
  \item{Install key -- 

\verb|sudo apt-get install emdebian-archive-keyring|}
  \item{Install Cross Compilers --
   \begin{verbatim}
   apt-get install                   \
   cpp-4.7-arm-linux-gnueabihf       \
   g++-4.7-arm-linux-gnueabihf       \
   gcc-4.7-arm-linux-gnueabihf       \
   gcc-4.7-arm-linux-gnueabihf-base  \
   gcc-4.7-plugin-dev-arm-linux-gnueabihf
   \end{verbatim}
   }
  \item{Install (not all needed? Note, some in backports):

  \begin{verbatim}
  libncurses-dev build-essential
  ccache git autoconf autoconf2.13
  gnu-standards libtool u-boot-tools
  debootstrap qemu qemu-user-static
  \end{verbatim}

  \item{Update everything to latest in wheezy-backports:
  \verb|apt-get -t wheezy-backports dist-upgrade|}
}


\verb|apt-get install ccache git|}
  \item{Add this to \verb|~/.bashrc| -- 


\verb|export PATH="/usr/lib/ccache:$PATH"|}

\item{Set up symlinks without version numbers (super crufty)

   \begin{verbatim}
   cd /usr/lib/ccache/
   sudo ln -s ../../bin/ccache arm-linux-gnueabihf-g++
   sudo ln -s ../../bin/ccache arm-linux-gnueabihf-gcc
   cd /usr/bin/
   sudo ln -s arm-linux-gnueabihf-cpp-4.7 arm-linux-gnueabihf-cpp
   sudo ln -s arm-linux-gnueabihf-g++-4.7 arm-linux-gnueabihf-g++
   sudo ln -s arm-linux-gnueabihf-gcc-4.7 arm-linux-gnueabihf-gcc
   sudo ln -s arm-linux-gnueabihf-gcc-ar-4.7 arm-linux-gnueabihf-gcc-ar
   sudo ln -s arm-linux-gnueabihf-gcc-nm-4.7 arm-linux-gnueabihf-gcc-nm
   sudo ln -s arm-linux-gnueabihf-gcc-ranlib-4.7 arm-linux-gnueabihf-gcc-ranlib
   sudo ln -s arm-linux-gnueabihf-gcov-4.7 arm-linux-gnueabihf-gcov
   \end{verbatim}
   }

\end{enumerate}


\begin{itemize}
  \item{fb}
  \item{X Windows}
  \item{EZTAZ}
  \item{OctoPrint Web Interface}
\end{itemize}

\subsection{OctoPrint}
OctoPrint main site: \url{http://www.octoprint.org}

OctoPrint uses Python 2.7.
Until OctoPrint is packaged in Debian, it will be built on a host system and 
installed to \verb|/usr/local|.

\begin{itemize}
  \item{\verb|adduser octo|}
  \item{\verb|su - octo|}
  \item{\verb|git clone git://github.com/foosel/OctoPrint.git|}
  \item{\verb|cd OctoPrint|}
  \item{\verb|git checkout 1.2.0-dev|}
  \item{This may be a good one to consider:

\verb|git checkout remotes/origin/devel|}
  \item{See what dependencies there are:

\verb|cat requirements.txt|}
\item{
\begin{verbatim}
apt-get install -t wheezy-backports python-jinja2 python-six  \
python-pygments python-docutils
\end{verbatim}
}

\item{As root, install unpackaged dependencies:
\begin{verbatim}
pip install -r requirements.txt
\end{verbatim}
}

\item{Install to \verb|/usr/local|:
\begin{verbatim}
python setup.py install
\end{verbatim}
}

\item{Set up \verb|/etc/default/octoprint|
\begin{verbatim}
OCTOPRINT_USER=octo
PORT=8080
DAEMON=/usr/local/bin/octoprint
DAEMON_ARGS="--port=$PORT"
UMASK=022
START=yes
\end{verbatim}
}

\item{Set up \verb|/etc/init.d/octoprint|
\begin{verbatim}
cp scripts/octoprint.init /etc/init.d/octoprint
chmod 755 /etc/init.d/octoprint
update-rc.d octoprint defaults
\end{verbatim}
}

\end{itemize}

OctoPrint 1.2.0-dev dependencies:

\begin{itemize}
\item{\verb|flask==0.9| -- Wheezy has \verb|python-flask 0.8|}
\item{\verb|werkzeug==0.8.3| -- Wheezy has \verb|python-werkzeug 0.8.3+dfsg-1|}
\item{\verb|tornado==3.0.2| -- Wheezy has \verb|python-tornado 2.3|}
\item{\verb|sockjs-tornado>=1.0.0|}
\item{\verb|PyYAML==3.10| -- Wheezy has \verb|python-yaml 3.10|}
\item{\verb|Flask-Login==0.2.2|}
\item{\verb|Flask-Principal==0.3.5|}
\item{\verb|pyserial>=2.6| -- Wheezy has \verb|python-serial 2.5|}
\item{\verb|netaddr>=0.7.10| -- Wheezy has \verb|python-netaddr 0.7.7|}
\item{\verb|mock>=1.0.1| -- Wheezy has \verb|python-mock 0.8.0|}
\item{\verb|nose>=1.3.0| -- Wheezy has \verb|python-nose 1.1.2|}
\item{\verb|sphinxcontrib-httpdomain|}
\item{\verb|sphinx_rtd_theme|}
\end{itemize}

Most python modules for OctoPrint 1.2.0-dev are in Debian, but they are
older versions. There may be newer versions that can be rebuilt from Jessie
and Sid.

These are in Wheezy:

\verb|apt-get install python2.7 python-werkzeug python-yaml python-pip|

These are pulled in by the dependencies of pip requirements.txt,
that are in Wheezy, but aren't explicitly named as requirements.

Set up the octo user with our OctoPrint config, stored at
\verb|/home/octo/.octoprint/config.yaml|

\verb|chown octo:octo /home/octo/.octoprint/config.yaml|

Change the \verb|key| as needed.

\begin{verbatim}
accessControl:
  enabled: false
api:
  enabled: true
  key: 19A7C56E31B74257955E49E5561D019D
appearance:
  color: yellow
  name: Easy TAZ Mini
cura: {}
feature:
  gCodeVisualizer: false
  sdSupport: false
  temperatureGraph: false
gcodeViewer: {}
printerParameters:
  bedDimensions:
    r: 100
    x: 150
    y: 150
  movementSpeed:
    x: 8500
    y: 7500
    z: 400
serial:
  autoconnect: true
  baudrate: 115200
  port: /dev/ttyACM0
  timeout:
    communication: 10.0
    connection: 10.0
    sdStatus: 2.0
server:
  firstRun: false
system: {}
temperature: {}
webcam:
  watermark: false
\end{verbatim}

\section{3D Object Processing}

\begin{itemize}
  \item{Slic3r}
  \item{Meshlab}
\end{itemize}

\section{Misc}

Adding "noswap" to the kernel boot command line should make it skip trying
to activate swap, but it doesn't in all cases. This hack:

\verb|echo "NOSWAP=yes" >> /etc/default/rcS|

Areas to fix to speed up boot:

\begin{itemize}
  \item{hotplug}
  \item{\verb|/dev| populated statically}
  \item{\verb|/tmp| is tmpfs, so no need to "clean" it}
\end{itemize}

